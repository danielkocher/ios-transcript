%%%%%%%%%%%%%%%%%%%%%%%%%%%%%%%%%%%%%%%%%%%%%%%%%%%%%%%%%%%%%%%%%%%%%%%%%%%%%%%%%%%%%%%%%%%%%%%%%%%%%%%%%%%%%%%%%%%%%%%%%%%%%%%%%%%%%%%%%%%%%%%%%%%%%%
% 20140112 - Introduction to Operating Systems VO
%%%%%%%%%%%%%%%%%%%%%%%%%%%%%%%%%%%%%%%%%%%%%%%%%%%%%%%%%%%%%%%%%%%%%%%%%%%%%%%%%%%%%%%%%%%%%%%%%%%%%%%%%%%%%%%%%%%%%%%%%%%%%%%%%%%%%%%%%%%%%%%%%%%%%%

%fancyhdr
\lhead{IOS VO} 
\rhead{2014-01-12}

%%%%%%%%%%%%%%%%%%%%%%%%%%%%%%%%%%%%%%%%%%%%%%%%%%%%%%%%%%%%%%%%%%%%%%%%%%%%%%%%%%%%%%%%%%%%%%%%%%%%%%%%%%%%%%%%%%%%%%%%%%%%%%%%%%%%%%%%%%%%%%%%%%%%%%

\par{
	\noindent\underline{Atomic instructions:}
	\par{
		\noindent
		Increment (\texttt{x = x + 1;})/Decrement (\texttt{x = x - 1;}) have to be atomic. Three ways to accomplish this atomicity:
		\parskip0pt\begin{enumerate}
			\item{\textit{compare \& swap} (CAS)}
			\item{\textit{load \& link} (LL)}
			\item{\textit{store condition} (SC)}
		\end{enumerate}
		The memory bus gets locked (implemented in hardware).
	}
}
\par{
	\noindent\underline{ABA problem:}
	\begin{figure}[H]
		\centering
		\begin{tikzpicture}		
			\node at (0, 0) (time_start) {};
			\foreach \i in {3, 6, 9}{
				\node[right = \i of time_start] (time_\i) {};
			}
			\node[right = 10 of time_start] (time_end) {};

			\draw[->, >=stealth] (time_start.center) -- (time_end.center) node[right] {\footnotesize{time}};
			\draw (time_start.north) -- (time_start.south);
			\foreach \i in {3, 6, 9}{
				\draw (time_\i.north) -- (time_\i.south);
			}

			\node[below right = 0.1 and 0.75 of time_start] {\footnotesize{lookup}};
			\node[above right = 0.1 and 0.5 of time_start] {\footnotesize{A $\rightsquigarrow$ m: A;}};

			\node[below right = 0.1 and 0.65 of time_3] {\footnotesize{no lookup}};
			\node[above right = 0.1 and 0.5 of time_3] {\footnotesize{B $\rightsquigarrow$ m: B;}};

			\node[below right = 0.1 and 0.75 of time_6] {\footnotesize{lookup}};
			\node[above right = 0.1 and 0.5 of time_6] {\footnotesize{A $\rightsquigarrow$ m: A;}};

		\end{tikzpicture}
		\caption{The ABA problem.}
		\label{fig:abaproblem}
	\end{figure}
	\par{
		\noindent
		Because of the ABA problem, lots of algorithms of the literature were not used any longer.
	}
	\par{
		\noindent
		How to avoid the ABA problem:
		\par{
			\noindent
			32-bit addresses; arbitrary 8 bits as version number; remaining 24 bits contain the data. \newline
			This solution does a good job but does not solve the problem completely because the 8 bit used as version number can overflow (already happened in the computational systems group).
		}
	}
}
\par{
	\noindent\underline{Spawn a process (C-like syntax \& semantics):}
	\begin{lstlisting}[language = C, frame = none]
	int pid;

	// make a new process; pid is the process identifier
	pid = fork();

	if(pid < 0)
		// error
	else if(pid > 0)
		// I am the parent process
	else
		// I am the child process
	\end{lstlisting}
}
\par{
	\noindent\underline{Wait for a process (C-like syntax \& semantics):}
	\begin{lstlisting}[language = C, frame = none]
	int pid;
	int status;

	// wait only returns when a child process terminates
	pid = wait(&status);

	if(pid <= 0)
		// error or there are not awaited child processes
	else
		// the child process pid has terminated with exit status
	\end{lstlisting}
}
\par{
	\noindent The pendant to \texttt{wait} is the \texttt{exit} system call:
	\begin{lstlisting}[language = C, frame = none]
	int status = 0;
	exit(status);
	\end{lstlisting}
}
\par{
	\noindent
	The consequence of using \texttt{wait}, \texttt{exit} and \texttt{fork} is the possible generation of \textit{zombie} processes. \newline
	If the parent processes exit, these are called \textit{orphans}. The root node adopts all orphans in the hierarchy.
}
\par{
	\noindent\underline{Booting the operating system:}
	\par{
		\noindent
		The BIOS defines how the operating system is loaded into the main memory (RAM). The CPU is a stupid machine which only fetches, decodes and executes instructions one-by-one. At some point of time, \texttt{fork} has to be called but there is not infrastructure available. \newline
		In the bootstraping process, everything is set up by hand (ready queue, et cetera). The \textit{init} process has pid 1, the \textit{idle} process has pid 0.
	}
	\begin{figure}[H]
		\centering
		\begin{tikzpicture}
			\node[draw, circle] at (5, 10) (kernel) {\footnotesize{kernel}};
			\node[left = 2 of kernel] (bios_ram) {\footnotesize{BIOS, RAM}};
			\draw[->, >=stealth] (kernel.west) -- (bios_ram.east);

			\node[below = 0.25 of bios_ram] (pc) {\footnotesize{pc}};
			\node[left = 2 of pc] (pc_left_invis) {};
			\node[left = 0.5 of bios_ram] (bios_ram_left_invis) {};
			\draw[->, >=stealth] (pc.west) .. controls (pc_left_invis) and (bios_ram_left_invis) .. (bios_ram.west);

			\node[below = 0.25 of pc] (booting) {\footnotesize{booting}};
			\draw (booting.south) -- ++(0, -0.25) -- ++(-1, 0) -- ++(0, -0.25) node[below, text width = 1cm, text centered] (idle) {\footnotesize{idle pid 0}};
			\draw (booting.south) -- ++(0, -0.25) -- ++(1, 0) -- ++(0, -0.25) node[below, text width = 1cm, text centered] (init) {\footnotesize{init pid 1}};

			\draw (idle.south) -- ++(0, -0.5) node[below] (idle_below_invis) {};
			\draw[->] (idle_below_invis.south east) arc (0:360: 2mm);

			\draw (init.south) -- ++(0, -0.5) node[below] (shell) {\footnotesize{shell}};
			\draw (init.south) ++(0, -0.25) -- ++(1, 0) -- ++(0, -0.25) node[below = 0.2] {\footnotesize{\ldots}};
		\end{tikzpicture}
		\caption{The boot process.}
		\label{fig:bootprocess}
	\end{figure}
	\par{
		\noindent
		The \texttt{init} process is the parent of all other processes (except the \texttt{idle} process which loops infinitely). When an application is called using the shell, e.g. \texttt{\textgreater ls}, the call is parsed. \newline
		How can the first process be spawned? \texttt{\textgreater ls\&} (this is the first fork).
	}
 }
 \par{
 	\noindent\underline{Virtualization:}
 	\par{
 		\noindent
 		\parskip0pt\begin{itemize}
 			\item{system virtualization, e.g. XEN, KVM, \ldots}
 			\item{process virtualization, e.g. JVM, \ldots}
 		\end{itemize}
 	}
 	\begin{figure}[H]
 		\centering
 		\begin{tikzpicture}
 			\tikzset{vm/.style = {draw, rounded corners, minimum height = 10cm, minimum width = 4cm}}

 			\node[vm] (vm1) {};
 			\node[vm, right = 0.75 of vm1] (vm2) {};
 			\draw (vm1.north east) ++(0.375, -0.25) -- ++(0, -8cm) node (eol_invis) {};
 			\node[draw, rounded corners, minimum height = 2cm, minimum width = 10cm, text width = 2cm, text centered, below = 2 of eol_invis] (hypervisor) {\footnotesize{VMM Hypervisor}};
 			\node[right = 1 of hypervisor.east] {\footnotesize{kernel}};

 			\node[below = 0 of vm1.north] (vm1_t) {\footnotesize{VM}};
 			\node[below = 0 of vm2.north] (vm2_t) {\footnotesize{VM}};

 			\draw (vm1.west) ++(-0.25, 0) node[left] (linux) {\footnotesize{Linux}} -- ++(9.25cm, 0) node[right] (osx) {\footnotesize{OS X}};
 			\draw[dashed] (vm1.west) ++(-0.25, -0.5) -- ++(9.25cm, 0);
 			\draw[->, >=stealth] (vm1.center) ++(0, 0.25) node[above right] {\footnotesize{user}} -- ++(0, -1.25) node[below right] {\footnotesize{user}};
 			\draw[->, >=stealth] (vm2.center) ++(0, 0.25) node[above right] {\footnotesize{user}} -- ++(0, -1.25) node[below right] {\footnotesize{user}};

 			\draw[->, >=stealth] (vm1.south) ++(0, 0.25) node[above] {\footnotesize{trap}} -- ++(0, -1.5);
 			\draw[->, >=stealth] (vm2.south) ++(0, 0.25) node[above] {\footnotesize{trap}} -- ++(0, -1.5);
 		\end{tikzpicture}
 		\caption{Virtualization.}
 		\label{fig:virtualization}
 	\end{figure}
 	\par{
 		\noindent
 		VMM \ldots Virtual Machine Monitor
 	}
 }

 \clearpage
 \par{
 	\noindent\underline{Self-compiling compiler:}
 	\par{
 		\noindent
 		$C^*$ compiler written in $C^*$: \texttt{cstar.c}.
 		\par{
 			\noindent
 			\texttt{\textgreater gcc cstar.c} $\rightsquigarrow$ \texttt{cstar} (for the x86 architecture) \newline
 			\texttt{\textgreater cstar cstar.c} $\rightsquigarrow$ \texttt{cstar1} (DLX, \texttt{cstar $\not=$ cstar1}) \newline
 			\texttt{\textgreater cstar1 cstar.c} $\rightsquigarrow$ \texttt{cstar2} (DLX, \texttt{cstar1 $=$ cstar2}) \newline
 	}
 	\par{
 		\noindent\underline{Accessing an array:}
 		\par{
 			\noindent
 			\texttt{$addr(a[i]) = addr(a) + i \cdot sizeof(elem)$}.
		}
		\begin{figure}[H]
			\centering
			\begin{tikzpicture}
				\tikzset{arrayelem/.style = {draw, minimum height = 1cm, minimum width = 1cm}}
				\tikzset{arraydata/.style = {draw, minimum height = 1cm, minimum width = 2cm}}

				\node[arrayelem] at (0, 0) (arrayelem_0) {}; 
				\node[below = 0 of arrayelem_0] (arrayelem_0_t) {\footnotesize{0}};
				\foreach \i [count = \ii from 0] in {1, 2, 3, 4}{
					\node[arrayelem, right = -0.015 of arrayelem_\ii] (arrayelem_\i) {};
					\node[below = 0 of arrayelem_\i] (arrayelem_\i_t) {\footnotesize{\i}};
				}
				\draw[<-, >=stealth] (arrayelem_0.north west) -- ++(0, 0.5) node[above] (array_t) {\footnotesize{array}};

				\node[arrayelem, right = 7 of array_t] (a) {\footnotesize{a}};
				\node[arraydata, right = -0.015 of a] (a_data) {\footnotesize{\ldots}}; 
				\node[arrayelem, below = -0.015 of a] (b) {\footnotesize{b}};
				\node[arraydata, right = -0.015 of b] (b_data) {\footnotesize{\ldots}}; 
				\node[arrayelem, below = -0.015 of b] (c) {\footnotesize{c}};
				\node[arraydata, right = -0.015 of c] (c_data) {\footnotesize{\ldots}}; 
				\draw[<-, >=stealth] (a_data.north east) -- ++(0.5, 0.5) node[right] {\footnotesize{obj}};
			\end{tikzpicture}
			\caption{Accessing an element of an array.}
			\label{fig:arrayaccess}
		\end{figure}
 	}
 }