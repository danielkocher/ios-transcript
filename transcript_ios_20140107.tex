%%%%%%%%%%%%%%%%%%%%%%%%%%%%%%%%%%%%%%%%%%%%%%%%%%%%%%%%%%%%%%%%%%%%%%%%%%%%%%%%%%%%%%%%%%%%%%%%%%%%%%%%%%%%%%%%%%%%%%%%%%%%%%%%%%%%%%%%%%%%%%%%%%%%%%
% 20140107 - Introduction to Operating Systems VO
%%%%%%%%%%%%%%%%%%%%%%%%%%%%%%%%%%%%%%%%%%%%%%%%%%%%%%%%%%%%%%%%%%%%%%%%%%%%%%%%%%%%%%%%%%%%%%%%%%%%%%%%%%%%%%%%%%%%%%%%%%%%%%%%%%%%%%%%%%%%%%%%%%%%%%

%fancyhdr
\lhead{IOS VO} 
\rhead{2014-01-07}

%%%%%%%%%%%%%%%%%%%%%%%%%%%%%%%%%%%%%%%%%%%%%%%%%%%%%%%%%%%%%%%%%%%%%%%%%%%%%%%%%%%%%%%%%%%%%%%%%%%%%%%%%%%%%%%%%%%%%%%%%%%%%%%%%%%%%%%%%%%%%%%%%%%%%%

\par{
	\noindent\underline{Recapitulation: memory of concurrent processes}
	\begin{figure}[H]
		\centering
		\begin{tikzpicture}
			\tikzset{memory_entry/.style = {draw, minimum width = 3cm}}
			
			\node[memory_entry, minimum height = 1.5cm] at (0, 0) (code_procA) {\footnotesize{Code}};
			\node[memory_entry, minimum height = 1cm, above = -0.015 of code_procA] (globals_procA) {\footnotesize{Globals / Strings}};
			\node[memory_entry, minimum height = 3.5cm, above = -0.015 of globals_procA] (stack_heap_procA) {};
			\draw[->, >=stealth] (stack_heap_procA.south) -- ++(0, 0.5) node[right] {\footnotesize{Heap}};
			\draw[->, >=stealth] (stack_heap_procA.north) -- ++(0, -0.5) node[right] {\footnotesize{Stack}};
			\draw[<-, >=stealth] (code_procA.east) -- ++(0.5, 0) node[right] {\footnotesize{PC}};
			\node[left = 0 of code_procA.south west] {\footnotesize{0}};
			\node[left = 0 of stack_heap_procA.north west] {\footnotesize{max}};
			\node[draw, minimum height = 0.5, minimum width = 0.5, above right = 0.5 of stack_heap_procA.south west] (heap_element) {};
			\node[below right = 0.5 of code_procA.north west] (code_invis) {};
			\node[left = 0.5 of globals_procA.south west] (code_invis_way1) {};
			\node[left = 0.5 of globals_procA.north west] (code_invis_way2) {};
			\draw[->, >=stealth] (code_invis) .. controls (code_invis_way1) and (code_invis_way2) .. (heap_element.west);
			\node[below = 0 of code_procA.south] {\footnotesize{process A}};

			\node[memory_entry, minimum height = 1.5cm, right = 4 of code_procA] (code_procB) {\footnotesize{Code}};
			\node[memory_entry, minimum height = 1cm, above = -0.015 of code_procB] (globals_procB) {\footnotesize{Globals / Strings}};
			\node[memory_entry, minimum height = 3.5cm, above = -0.015 of globals_procB] (stack_heap_procB) {};
			\draw[->, >=stealth] (stack_heap_procB.south) -- ++(0, 0.5) node[right] {\footnotesize{Heap}};
			\draw[->, >=stealth] (stack_heap_procB.north) -- ++(0, -0.5) node[right] {\footnotesize{Stack}};
			\node[left = 0 of code_procB.south west] {\footnotesize{0}};
			\node[left = 0 of stack_heap_procB.north west] {\footnotesize{max}};
			\node[below = 0 of code_procB.south] {\footnotesize{process B}};

			\draw (stack_heap_procA.east) ++(2, 0) -- ++(0, -2);
			\draw (stack_heap_procA.east) ++(2.1, 0) -- ++(0, -2);
		\end{tikzpicture}
		\caption{Concurrent processes and their memory layout}
		\label{fig:concurrprocsmemlayout}
	\end{figure}
}

\par{
	\noindent
	Fundamental reason for parallelism/concurrency (parallelism $\not=$ concurrency!): \newline
	Because the world we are living in consists of lots of parallel processes $\Rightarrow$ real world picture; it is not about convenience! It is also about correctness proofs $\Rightarrow$ to prove that one big program is correct is much more difficult to show than multiple concurrent programs. \newline
	The ultimate prerequisite to accomplish this: each process has to think he owns the whole machine \& memory.
}

\par{
	\noindent
	Physical memory is:
	\parskip0pt\begin{enumerate}
		\item{a set of addresses}
		\item{content of these }
	\end{enumerate}
	Virtual memory is:
	\parskip0pt\begin{enumerate}
		\item{a set of addresses}
	\end{enumerate}
	Paging: Split virtual and physical memory into 4kB pages (of addresses).
	\begin{figure}[H]
		\centering
		\begin{tikzpicture}
			\tikzset{memory_entry/.style = {draw, minimum width = 3cm, minimum height = 0.5cm}}
			\tikzset{page_table_entry/.style = {draw, minimum width = 2cm, minimum height = 0.5cm}}

			\node[memory_entry] at (0, 0) (virtmem_entry1) {};
			\node[memory_entry, above = -0.015 of virtmem_entry1] (virtmem_entry2) {};
			\node[draw, minimum width = 3cm, minimum height = 3cm, above = -0.015 of virtmem_entry2] (virtmem_dots) {\footnotesize{\ldots}};
			\node[memory_entry, above = -0.015 of virtmem_dots] (virtmem_entry3) {};
			\node[memory_entry, above = -0.015 of virtmem_entry3] (virtmem_entry4) {};
			\node[below = 0 of virtmem_entry1] (virtmem_t) {\footnotesize{virtual memory}};
			\draw[<-, >=stealth] (virtmem_entry1.north west) -- ++(-0.5, 0) node[left] {\footnotesize{4kB}};
			\draw[<-, >=stealth] (virtmem_entry2.north west) -- ++(-0.5, 0) node[left] {\footnotesize{8kB}};

			\node[page_table_entry, right = 2 of virtmem_entry2] (pt_entry1) {};
			\node[page_table_entry, above = -0.015 of pt_entry1] (pt_entry2) {};
			\node[draw, minimum width = 2cm, minimum height = 1cm, above = -0.015 of pt_entry2] (pt_dots) {\footnotesize{\ldots}};
			\node[page_table_entry, above = -0.015 of pt_dots] (pt_entry3) {};
			\node[page_table_entry, above = -0.015 of pt_entry3] (pt_entry4) {};
			\node[below = 0 of pt_entry1] (pt_t) {\footnotesize{page table}};

			\node[memory_entry, right = 6 of virtmem_entry1] (physmem_entry1) {};
			\node[memory_entry, above = -0.015 of physmem_entry1] (physmem_entry2) {};
			\node[draw, minimum width = 3cm, minimum height = 3cm, above = -0.015 of physmem_entry2] (physmem_dots) {\footnotesize{\ldots}};
			\node[memory_entry, above = -0.015 of physmem_dots] (physmem_entry3) {};
			\node[memory_entry, above = -0.015 of physmem_entry3] (physmem_entry4) {};
			\node[below = 0 of physmem_entry1] (physmem_t) {\footnotesize{physical memory}};
			\draw[<-, >=stealth] (physmem_entry1.north east) -- ++(0.5, 0) node[right] {\footnotesize{4kB}};
			\draw[<-, >=stealth] (physmem_entry2.north east) -- ++(0.5, 0) node[right] {\footnotesize{8kB}};

			\foreach \i in {1, 2} {
				\node[right = 0.5 of virtmem_entry\i] (virtmem_entry\i_invis_right) {};
				\node[left = 0.5 of pt_entry\i] (pt_entry\i_invis_left) {};
				\draw[->, >=stealth] (virtmem_entry\i.east) .. controls (virtmem_entry\i_invis_right) and (pt_entry\i_invis_left) .. (pt_entry\i.west);
			}

			\node[right = 0.5 of pt_entry1] (pt_entry1_invis_right) {};
			\node[left = 0.5 of physmem_entry3] (physmem_entry3_invis_left) {};
			\draw[->, >=stealth] (pt_entry1.east) .. controls (pt_entry1_invis_right) and (physmem_entry3_invis_left) .. (physmem_entry3.west);

			\node[right = 0.5 of pt_entry2] (pt_entry2_invis_right) {};
			\node[left = 0.5 of physmem_entry1] (physmem_entry1_invis_left) {};
			\draw[->, >=stealth] (pt_entry2.east) .. controls (pt_entry2_invis_right) and (physmem_entry1_invis_left) .. (physmem_entry1.west);
		\end{tikzpicture}
		\caption{Mapping virtual $\leftrightarrow$ physical memory}
		\label{fig:mappingvirtphysmem}
	\end{figure}
}