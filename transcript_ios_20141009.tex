%%%%%%%%%%%%%%%%%%%%%%%%%%%%%%%%%%%%%%%%%%%%%%%%%%%%%%%%%%%%%%%%%%%%%%%%%%%%%%%%%%%%%%%%%%%%%%%%%%%%%%%%%%%%%%%%%%%%%%%%%%%%%%%%%%%%%%%%%%%%%%%%%%%%%%
% 20141009 - Introduction to Operating Systems VO
%%%%%%%%%%%%%%%%%%%%%%%%%%%%%%%%%%%%%%%%%%%%%%%%%%%%%%%%%%%%%%%%%%%%%%%%%%%%%%%%%%%%%%%%%%%%%%%%%%%%%%%%%%%%%%%%%%%%%%%%%%%%%%%%%%%%%%%%%%%%%%%%%%%%%%

%fancyhdr
\lhead{IOS VO} 
\rhead{2014-10-09}

%%%%%%%%%%%%%%%%%%%%%%%%%%%%%%%%%%%%%%%%%%%%%%%%%%%%%%%%%%%%%%%%%%%%%%%%%%%%%%%%%%%%%%%%%%%%%%%%%%%%%%%%%%%%%%%%%%%%%%%%%%%%%%%%%%%%%%%%%%%%%%%%%%%%%%

\par{
	\noindent
	What we want:
	\begin{figure}[H]
		\centering
		\begin{tikzpicture}
			\node[draw, minimum width = 2cm, minimum height = 8cm] (physmem) at (0, 0) {};
			\node[left = 0 of physmem.south west] (physmem_0) {0};
			\node[left = 0 of physmem.north west] (physmem_max) {max};
			\node[below = 0.25 of physmem.south] (physmem_t) {phys. memory};

			\foreach \i [count = \ii from 1] in {A, B}{
				\node[draw, right = \ii * 4 of physmem.north east, minimum width = 0.25cm, minimum height = 0.5cm] (reg_proc\i) {};
				\node[above = 0 of reg_proc\i] (reg_proc\i_t) {reg};
				\node[draw, right = 0.25 of reg_proc\i.east, minimum width = 2cm, minimum height = 0.5cm] (stack_proc\i) {\footnotesize{Stack}};
				\node[draw, below = 1.5 of stack_proc\i, minimum width = 2cm, minimum height = 0.5cm] (heap_proc\i) {\footnotesize{Heap}};
				\node[draw, below = -0.015 of heap_proc\i.south, minimum width = 2cm, minimum height = 1cm] (globals_proc\i) {\footnotesize{Globals}};
				\node[draw, below = -0.015 of globals_proc\i.south, minimum width = 2cm, minimum height = 3cm] (code_proc\i) {\footnotesize{Code}};
				\draw (stack_proc\i.north west) -- (stack_proc\i.north east) -- (code_proc\i.south east) -- (code_proc\i.south west) -- (stack_proc\i.north west);
				\node[below = 0.25 of code_proc\i.south] (proc\i_t) {process \i};
				\draw[<-, >=stealth] (code_proc\i.west) -- ++(-0.5, 0);
				\draw[->, >=stealth] (stack_proc\i.south) -- ++(0, -0.25);
				\draw[->, >=stealth] (heap_proc\i.north) -- ++(0, 0.25);
				\draw[dotted] (code_proc\i.west) ++(-0.25, 0) -- ++(0, -1.75);
			}
			\draw[dotted] (code_procA.west) ++(-0.25, -1.75) -- ++(4, 0);
			\draw[<-, >=stealth] (code_procB.west) ++(-0.75, -1.75) -- ++(0, -1) node[below] {\footnotesize{context switch}};
		\end{tikzpicture}
		\caption{Each process thinks it owns the whole memory.}
		\label{fig:concurrencymem}
	\end{figure}
	\noindent
	Each process has its own virtual memory which is loaded whenever a context switch is performed. \textit{Concurrency} (competing for resources, time sharing) $\not=$ \textit{Parallelism} (truely parallel, multiple processors). \textit{Spatial isolation} is necessary to enable concurrency.
}

\par{
	\noindent\underline{Data structures:}
	\par{
		\noindent
		2 Goals:
		\parskip0pt\begin{enumerate}
			\item{Abstraction: table, list, tree, graph, \ldots}
			\item{Safety: only accessing memory we have declared, memory access as specified.}
		\end{enumerate}
	}
	\par{
		\noindent
		2 levels where we can have safety:
		\begin{figure}[H]
			\centering
			\begin{tikzpicture}
				\node[draw, minimum height = 2cm, minimum width = 2cm, text width = 2cm, text centered] (gc) at (1, 0) {Garbage Collector};
				\node[draw, right = 1 of gc, minimum height = 1cm, minimum width = 4cm] (range_checking) {range checking};
				\draw[] (gc.north west) ++(-0.5, 0.5) -- ++(12, 0);
				\node[draw, above = 2 of range_checking, minimum height = 1cm, minimum width = 4cm] (type_checking) {type checking};

				\node[right = 2 of range_checking] (runtime_t) {runtime};
				\node[right = 2 of type_checking] (compiletime_t) {compile time};

				\draw[<-, >=stealth] (type_checking.north east) -- ++(0.5, 0.5) node[right, text width = 4cm] {\footnotesize{has to be strong enough, \textit{static typing}}};
			\end{tikzpicture}
			\caption{safety levels}
			\label{fig:safetylevels}
		\end{figure}
	}
	\par{
		\noindent
		Unless all accesses use constants as indices, in general no compiler can perform a range check at compile time. At some point there is a limit. E.g. \texttt{i = 10; a[i];} is recognized by some compilers. \newline
		If you do not allow dynamic allocation, your language does not allow to write a program which accesses memory outside (type and range checks required of course!). \newline
		Suppose a \texttt{free} mechanism: you could free something and access it afterwards, or you could do \texttt{free(obj);} which creates a \textit{dangling pointer}, and do \texttt{obj.field = 1;} afterwards.
	}
	\par{
		\noindent
		\begin{minipage}{0.45\textwidth}
			\setlength{\parskip}{12pt plus2pt minus2pt}
			\par{
				\noindent\underline{Arrays:}
				\begin{lstlisting}[language = C, frame = none]
	int a[3];
				\end{lstlisting}
				The advantage of the contiguous layout in memory: constant time access.\newline
				The big drawback of the contiguous layout in memory: fragmentation. \newline
				3 ingredients of fragmentation:
				\parskip0pt\begin{enumerate}
					\item{contiguously allocated memory gives us constant time access (could be splitted as well)}
					\item{blocks of different size}
					\item{the order of deallocation is not equal to the order of allocation}
				\end{enumerate}
				In the worst case, there is a 50/50 allocation present. If we then want to allocate a big chunk of memory, it is not possible without some reordering.
			}
			\par{
				\noindent\underline{Records:}
				\begin{lstlisting}[language = C, frame = none]
	struct rec {
		int i;
		struct rec *r;
	};
				\end{lstlisting}
				The advantages of the contiguous layout in memory:
				\parskip0pt\begin{itemize}
					\item{constant time access (even faster than the array access because the offsets can be computed at compile time)}
					\item{no fragmentation (if the implementer knows what s/he is doing)}
				\end{itemize}
				The disadvantage is the fact that the access complexity is not constant but linear in the size of the list.
			}
		\end{minipage}%
		\hfill
		\begin{minipage}{0.45\textwidth}
			\begin{figure}[H]
				\centering
				\begin{tikzpicture}
					\def\cellheight{0.65cm}
					\tikzset{memcontcell/.style = {draw, minimum height = \cellheight, minimum width = 2cm}};
					\tikzset{memposcell/.style = {minimum height = \cellheight}};
					\tikzset{brace/.style = {decoration = {brace, mirror, raise = 4ex}, decorate}}

					\node[memposcell] (mem_pos0) at (0, 0) {0};
					\node[memcontcell, right = -0.015 of mem_pos0] (mem_cont0) {\texttt{3 (= i)}};

					\foreach \i [count = \ii from 0] in {1, ..., 8}{
						\node[memposcell, above = -0.015 of mem_pos\ii] (mem_pos\i) {\i};
					}

					\node[memcontcell, above = -0.015 of mem_cont0] (mem_cont1) {\texttt{3 (= r)}};
					\node[memcontcell, above = -0.015 of mem_cont1] (mem_cont2) {};
					\node[memcontcell, above = -0.015 of mem_cont2] (mem_cont3) {\texttt{4}};
					\node[memcontcell, above = -0.015 of mem_cont3] (mem_cont4) {\texttt{0}};
					\node[memcontcell, above = -0.015 of mem_cont4] (mem_cont5) {};
					\node[memcontcell, above = -0.015 of mem_cont5] (mem_cont6) {\texttt{5}};
					\node[memcontcell, above = -0.015 of mem_cont6] (mem_cont7) {\texttt{2}};
					\node[memcontcell, above = -0.015 of mem_cont7] (mem_cont8) {\texttt{6}};

					\node[draw, minimum height = 5cm, minimum width = 2cm, above = -0.015 of mem_cont8] (mem_contdots) {\ldots};
					\node[memcontcell, above = -0.015 of mem_contdots] (mem_contmax) {};
					\node[memposcell, left = -0.015 of mem_contmax] (mem_posmax) {max};
					\node[above = 0 of mem_contmax] (abstract_memory_t) {abstract memory};

					\draw[->, >=stealth] (mem_cont1.east) -- ++(0.5, 0) -- ++(0, \cellheight * 2) -- (mem_cont3.east);
					\draw[brace] (mem_cont0.south east) -- node[right, xshift = 6ex] {\texttt{struct rec}} (mem_cont1.north east);
					\draw[brace] (mem_cont3.south east) -- node[right, xshift = 6ex] {\texttt{struct rec}} (mem_cont4.north east);
					\draw[brace] (mem_cont6.south east) -- node[right, xshift = 6ex] {\texttt{int a[3]}} (mem_cont8.north east);
					\draw[<-, >=stealth, dotted] (mem_cont4.east) -- ++(1, 0.75) node[right] {zero: end of linked list};
				\end{tikzpicture}
				\caption{Arrays and records in the memory}
				\label{fig:arraysrecordsmem}
			\end{figure}
		\end{minipage}%
	}
	\clearpage
	\par{
		\noindent
		The only other option (to use contiguous allocation):
		\begin{center}
			\textbf{Construct data structures out of pointers}.
		\end{center}
		If we just allocate a billion of \texttt{struct rec} (which actually all have the same size), there is no fragmentation.
	}
}

\par{
	\noindent\underline{Memory Hierarchy:}
	\begin{figure}[H]
		\centering
		\begin{tikzpicture}	
			\tikzset{pyramidenode/.style = {draw, trapezium, trapezium left angle = 60, trapezium right angle = 60, minimum height = 1cm, text centered}}
			\node (throughput_t) at (0, 0) {\textit{throughput}};
			\node[above = 0.15 of throughput_t.north] (min_t) {min};
			\node[above = 6 of min_t] (max_t) {max};
			\draw[->, >=stealth] (min_t.north east) -- (max_t.south east);

			\node[pyramidenode, above right = 0 and 4 of min_t, text width = 3.574cm] (cloud) {Cloud};
			\node[pyramidenode, above = -0.015 of cloud, text width = 3.35cm] (storage) {Storage};
			\node[pyramidenode, above = -0.015 of storage, text width = 2.6cm] (mainmemory) {Main Memory}; 
			\node[pyramidenode, above = -0.015 of mainmemory, text width = 1.6cm] (caches) {Caches};
			\node[pyramidenode, above = -0.015 of caches, text width = 1.1125cm] (register) {Reg};
			\node[draw, above = -0.015 of register, regular polygon, regular polygon sides = 3, minimum width = 1cm, text width = 0.6cm] {};

			% invisible nodes to curve pointers
			\node[left = 1 of caches.west] (caches_left) {};
			\node[left = 0.5 of mainmemory.west] (mainmemory_left) {};
			\draw[<->, >=stealth] (caches.west) .. controls (caches_left) and (mainmemory_left) .. (mainmemory.west) node[above left = 0.25 and 0.4] {\tiny{Hardware}};

			\node[right = 1.5 of register.east] (register_right) {};
			\node[right = 0.5 of mainmemory.east] (mainmemory_right) {};
			\draw[<->, >=stealth] (register.east) .. controls (register_right) and (mainmemory_right) .. (mainmemory.east) node[above right = 0.75 and 0.4] {\tiny{Compiler/Interpreter}};

			\node[right = 0.5 of storage.east] (storage_right) {};
			\draw[->, >=stealth] (mainmemory.east) .. controls (mainmemory_right) and (storage_right) .. (storage.east) node[above right = 0.25 and 0.4] {\tiny{OS}};

			\node[right = 0.5 of cloud.east] (cloud_right) {};
			\draw[->, >=stealth] (storage.east) .. controls (storage_right) and (cloud_right) .. (cloud.east) node[above right = 0.25 and 0.4] {\tiny{Services}};

			\node[right = 0.5 of mainmemory.east] (memorymanagement) {\tiny{Memory management}};

			\node[right = 8.5 of throughput_t] (latency_t) {\textit{latency}};
			\node[above = 0.15 of latency_t.north] (lat_max_t) {max};
			\node[above = 6 of lat_max_t] (lat_min_t) {min};
			\draw[<-, >=stealth] (lat_max_t.north east) -- (lat_min_t.south east);

			\node[right = 0.25 of latency_t] (capacity_t) {\textit{capacity}};
			\node[above = 0.15 of capacity_t.north] (cap_max_t) {max};
			\node[above = 6 of cap_max_t] (cap_min_t) {min};
			\draw[<-, >=stealth] (cap_max_t.north east) -- (cap_min_t.south east);

			\node[right = 0.25 of capacity_t] (volatility_t) {\textit{volatility}};
			\node[above = 0.15 of volatility_t.north] (vol_min_t) {min};
			\node[above = 6 of vol_min_t] (vol_max_t) {max};
			\draw[->, >=stealth] (vol_min_t.north east) -- (vol_max_t.south east);
 		\end{tikzpicture}
 		\caption{Memory hierarchy}
 		\label{fig:memhierarchy}
	\end{figure}
	\par{
		\noindent
		\textit{Services}: Dropbox, Google Cloud, \ldots
	}
}