%%%%%%%%%%%%%%%%%%%%%%%%%%%%%%%%%%%%%%%%%%%%%%%%%%%%%%%%%%%%%%%%%%%%%%%%%%%%%%%%%%%%%%%%%%%%%%%%%%%%%%%%%%%%%%%%%%%%%%%%%%%%%%%%%%%%%%%%%%%%%%%%%%%%%%
% 20141010 - Introduction to Operating Systems VO
%%%%%%%%%%%%%%%%%%%%%%%%%%%%%%%%%%%%%%%%%%%%%%%%%%%%%%%%%%%%%%%%%%%%%%%%%%%%%%%%%%%%%%%%%%%%%%%%%%%%%%%%%%%%%%%%%%%%%%%%%%%%%%%%%%%%%%%%%%%%%%%%%%%%%%

%fancyhdr
\lhead{IOS VO} 
\rhead{2014-10-10}

%%%%%%%%%%%%%%%%%%%%%%%%%%%%%%%%%%%%%%%%%%%%%%%%%%%%%%%%%%%%%%%%%%%%%%%%%%%%%%%%%%%%%%%%%%%%%%%%%%%%%%%%%%%%%%%%%%%%%%%%%%%%%%%%%%%%%%%%%%%%%%%%%%%%%%

\par{
	\noindent\underline{Real-time application:}
	\par{\noindent You know how much time a specific operation takes and there is a guarantee on this max. time.}

	\noindent\underline{Thread Local Allocation Buffer (TLAB):}
	\par{
		\noindent
		Each thread has its own TLAB to allow faster allocation. Whenever a new object is allocated on the heap, the object will first be placed in the TLAB. The performance improvement is because the threads can allocate additional memory within the TLAB without locking anything. The TLAB is pre allocated for each thread.
	}
}

\par{
	\noindent\underline{Virtual Memory:}
	\par{
		\noindent
		\begin{figure}[H]
			\centering
			\begin{tikzpicture}
				\tikzset{memory/.style = {draw, minimum height = 6cm, minimum width = 2cm}}

				\node[memory] at (0, 0) (physical_memory) {};
				\node[right = 0 of physical_memory.south east] {\footnotesize{0}};
				\node[right = 0 of physical_memory.north east] {\footnotesize{max}};
				\node[below = 0.25 of physical_memory.south] {\footnotesize{physical memory}};

				\node[memory, right = 2 of physical_memory.east] (memory_procA) {}; 
				\node[below = 0.25 of memory_procA.south] (memory_procA_t) {\footnotesize{Process A}};
				\node[below = 0.25 of memory_procA_t.south, text width = 1.75cm] (code_mem_procA_t) {\footnotesize{code + memory + registers}};
				\draw[<-, >=stealth] (code_mem_procA_t.south) -- ++(0, -0.5) node[below] {\footnotesize{current}};

				\node[right = 0.75 of memory_procA.east] {\footnotesize{\ldots}};

				\node[memory, right = 2 of memory_procA.east] (memory_procB) {}; 
				\node[below = 0.25 of memory_procB.south] (memory_procB_t) {\footnotesize{Process B}};
				\node[below = 0.25 of memory_procB_t.south, text width = 1.75cm] (code_mem_procB_t) {\footnotesize{code + memory + registers}};

				\node[right = 0.75 of memory_procB.east] {\footnotesize{\ldots}};
				\node[right = 2 of memory_procB_t.east] {\footnotesize{Process $n$}};
			\end{tikzpicture}
			\caption{Virtual memory: basic idea}
			\label{fig:virtmem_basicidea}
		\end{figure}
	}
	\par{
		\noindent
		\begin{figure}[H]
			\centering
			\begin{tikzpicture}
				\tikzset{memory/.style = {draw, minimum height = 6cm, minimum width = 2cm}}
				\tikzset{brace/.style = {decoration = {brace, mirror, raise = 1ex}, decorate}}

				\node[memory] at (0, 0) (memory_procA) {};
				\node[below = 0.25 of memory_procA.south, text width = 2.5cm, text centered] (memory_procA_t) {\footnotesize{virtual address space of process A}};
				\draw (memory_procA_t.south west) -- (memory_procA_t.south east) node[below left = 0 and 0.5] {\footnotesize{Namespace}};
				\node[left = 0 of memory_procA.north west] {\footnotesize{$2^{64} -1$}};
				\node[left = 0 of memory_procA.south west] {\footnotesize{$0$}};
				\draw (-1, -2.25) -- ++(2, 0);
				\draw (-1, -1.5) -- ++(2, 0);
				\node at (0, -2.6) (first4kb) {\footnotesize{4kB}};
				\node at (0, -1.85) (second4kb) {\footnotesize{4kB}};
				\draw[<-, >=stealth] (-1, -1.85) -- ++(-0.5, 0) node[left] {\footnotesize{Page}};

				\node[draw, minimum height = 1cm, minimum width = 0.5cm, right = 2 of memory_procA] (page_table) {};
				\node[below = 0.25 of page_table.south, text width = 1cm, text centered] (page_table_t) {\footnotesize{page table}};
				\draw (3.02, -0.25) node (page_table_entry) {} -- ++(0.5, 0);
				\node[right = 1 of first4kb] (invis_first4kb) {};
				\node[left = 0.75 of page_table.south west] (invis_page_table_left) {};
				\draw[->, >=stealth] (memory_procA.south east) ++(0, 0.25) .. controls (invis_first4kb) and (invis_page_table_left) .. (page_table_entry.south);

				\node[draw, minimum height = 3cm, minimum width = 1.5cm, right = 1 of page_table] (physical_memory) {};
				\node[below = 0.25 of physical_memory.south, text width = 1.5cm, text centered] (physical_memory_t) {\footnotesize{physical memory}};
				\node[draw, minimum width = 1.5cm, minimum height = 0.5cm, left = -1.5135 of physical_memory] (physical_memory_entry) {\footnotesize{Code}};
				\node[right = 0.25 of page_table_entry.south east] (invis_page_table_right) {};
				\node[left = 0.25 of physical_memory_entry] (invis_physical_memory_left) {};
				\draw[o->, >=stealth] (page_table_entry.south east) ++(0, 0.025) .. controls (invis_page_table_right) and (invis_physical_memory_left) .. (physical_memory_entry.west);
				\node[left = 0 of physical_memory.south west] {\footnotesize{$0$}};
				\node[left = 0 of physical_memory.north west] {\footnotesize{$2^{32} - 1$}};
				\draw[brace] (physical_memory_entry.south east) -- node[right = 0.25, text width = 1cm, text centered] (brace_middle) {\footnotesize{4kB page frame}} (physical_memory_entry.north east);
				\draw[<-, >=stealth] (physical_memory.north) -- ++(-0.5, 1.5) node[left, text width = 2cm, text centered] {\footnotesize{Page frame allocator}};

				\node[memory, right = 2 of physical_memory] (memory_procB) {};
				\node[below = 0.25 of memory_procB.south, text width = 2.5cm, text centered] (memory_procB_t) {\footnotesize{virtual address space of process B}};
				\node[left = 0 of memory_procB.north west] {\footnotesize{$2^{64} -1$}};
				\node[left = 0 of memory_procB.south west] {\footnotesize{$0$}};
			\end{tikzpicture}
			\caption{Virtual memory: virtual and physical address space}
			\label{fig:virtmem_virtphysaddrspace}
		\end{figure}
		\par{
			\noindent
			Logical the address space is from $0$ to $2^{64} - 1$. But current hardware supports up to $2^{48}$ and current software supports only up to $2^{45}$ addresses.
		}
		\par{
			\noindent
			The page table resolves the mapping from virtual memory (only addresses) to physical memory (actual memory we can store data to). \newline
			Why do we use 4kB chunks? The smaller the pages the bigger the page table, e.g. if we would use 1B pages the page table would have a size of $2^{64}$ entries. The use of 4kB pages reduces this by a factor of 4.000. The fact that all pages have the same size (4kB) results in a big advantage: there is no fragmentation.
		}
	}
	\par{
		\noindent\underline{Memory Management Unit (MMU):}
		\par{
			\noindent
			Very fast hardware to resolve the mapping virtual address $\leftrightarrow$ physical address. It also serves as a memory protector and tells the operating systems which virtual addresses are accessible. Because there are millions of requests, this mechanism has to be very fast and thus is implemented in hardware.
		}
	}
	\par{
		\noindent\underline{Page fault:}
		\par{\noindent Page fault handler $\leftrightarrow$ Trap $\leftrightarrow$ MMU.}
		\begin{figure}[H]
			\centering
			\begin{tikzpicture}
				\tikzset{memory/.style = {draw, minimum height = 6cm, minimum width = 2cm}}
				\tikzset{brace/.style = {decoration = {brace, raise = 1ex}, decorate}}

				\node[memory] at (0, 0) (memory_proc) {};
				\node[below = 0 of memory_proc] (memory_proc_t) {\footnotesize{process}};
				\node[draw, minimum height = 0.5cm, minimum width = 2cm, left = -2.015 of memory_proc] (memory_proc_entry) {};
				\draw[brace] (memory_proc_entry.south west) -- node[left = 0.25, text width = 1cm, text centered] (brace_middle) {\footnotesize{4kB}} (memory_proc_entry.north west);

				\node[draw, minimum height = 3cm, minimum width = 2cm, right = 2 of memory_proc] (page_table) {};
				\node[below = 0 of page_table] (page_table_t) {\footnotesize{page table}};
				\node[draw, minimum height = 0.5cm, minimum width = 2cm, left = -2.015 of page_table] (page_table_entry) {};
				\draw[<-, >=stealth] (page_table.north) -- ++(0, 0.5) node[above] {\footnotesize{MMU}};
				\draw[o->, >=stealth] (memory_proc_entry.center) -- (page_table_entry.west);
				\draw[o->, >=stealth] (page_table_entry.center) -- ++(1.5, 0) node[right] {\footnotesize{?}};

				\node[memory, right = 2 of page_table] (memory_physical) {};
				\node[below = 0 of memory_physical] (memory_physical_t) {\footnotesize{physical memory}};
				\node[draw, minimum height = 0.5cm, minimum width = 2cm, left = -2.015 of memory_physical] (memory_physical_entry) {};

				\draw[->, >=stealth] (page_table_t.south) ++(0, -2) node[below right] {\footnotesize{MMU}} -- ++(0, -1);

				\node[memory, below = 3 of memory_proc] (memory_proc2) {};
				\node[below = 0 of memory_proc2] (memory_proc2_t) {\footnotesize{process}};
				\node[draw, minimum height = 0.5cm, minimum width = 2cm, left = -2.015 of memory_proc2] (memory_proc_entry2) {};
				\draw[brace] (memory_proc_entry2.south west) -- node[left = 0.25, text width = 1cm, text centered] (brace_middle) {\footnotesize{4kB}} (memory_proc_entry2.north west);

				\node[draw, minimum height = 3cm, minimum width = 2cm, right = 2 of memory_proc2] (page_table2) {};
				\node[below = 0 of page_table2] (page_table2_t) {\footnotesize{page table}};
				\node[draw, minimum height = 0.5cm, minimum width = 2cm, left = -2.015 of page_table2] (page_table_entry2) {};
				\draw[<-, >=stealth] (page_table2.north) -- ++(0, 0.5) node[above] {\footnotesize{MMU}};
				\draw[o->, >=stealth] (memory_proc_entry2.center) -- (page_table_entry2.west);
				
				\node[memory, right = 2 of page_table2] (memory_physical2) {};
				\node[below = 0 of memory_physical2] (memory_physical2_t) {\footnotesize{physical memory}};
				\node[draw, minimum height = 0.5cm, minimum width = 2cm, left = -2.015 of memory_physical2] (memory_physical_entry2) {\footnotesize{X}};
				\draw[o->, >=stealth] (page_table_entry2.center) -- (memory_physical_entry2.west);
			\end{tikzpicture}
			\caption{Resolving a page fault.}
			\label{fig:pagefaultresolve}
		\end{figure}
		\par{
			\noindent
			If a page fault occurs, the MMU resolves the corresponding mapping and the requested operation is executed again after the mapping was done by the MMU. This is called \textit{on-demand paging}.
		}
		\par{
			\noindent
			$\frac{2^{64}}{4kB} = \frac{2^{64}}{2^{12}} = 2^{52}$ is the size of the page table if 4kB pages are used. We would not be able to allocate such a big page table. \newline
			The address space is gigantic and the whole address space is (almost) never used completely by a single process (\textit{sparsely/tensed populated}). \newline
			Thus, size of the page table $\ll 2^{52} \Rightarrow$ no array is used $\Rightarrow$ \textit{2-level tree} is used.
		}
		\begin{figure}
			\centering
			\begin{tikzpicture}
				\tikzset{pd_entry/.style = {draw, minimum height = 1cm, minimum width = 1cm}}
				\tikzset{pt_entry/.style = {draw, minimum height = 0.5cm, minimum width = 0.5cm}}

				\node[pd_entry] (pd_entry1) {};
				\foreach \i [count = \ii from 1] in {2, 3, 4, 5, 6}{
					\node[pd_entry, right = -0.015 of pd_entry\ii] (pd_entry\i) {};
				}
				\node[right = 0 of pd_entry6.east, text width = 2cm, text centered] (pd_t) {\footnotesize{page directory}};

				\node[pt_entry, below left = 2 and 2 of pd_entry1] (pt_entry1) {};
				\foreach \i [count = \ii from 1] in {2, 3, 4, 5, 6, 7, 8}{
					\node[pt_entry, right = -0.015 of pt_entry\ii] (pt_entry\i) {};
				}
				\node[right = 0 of pt_entry8.east, text width = 2cm, text centered] (pt_t) {\footnotesize{page table}};

				\draw[->, >=stealth] (pd_entry1.center) -- (pt_entry4.north);
				\foreach \i in {1, 2, 3, 4, 5, 6, 7, 8}{
					\draw[->, >=stealth] (pt_entry\i.center) -- ++(0, -0.5);
				}

				\node[below = 1 of pd_entry6, text width = 4cm, text centered] {\footnotesize{constant time access, because the depth of the tree is known (2).}};
			\end{tikzpicture}
			\caption{2-level tree (or 2-phase tree).}
			\label{fig:2leveltree}
		\end{figure}
		\par{\noindent\textit{Prefetching} is even smarter. If a page is loaded, its neighbors (the page below/above it) are loaded within the same request (see spatial locality).}
		\par{
			\noindent
			Physical memory full on page fault $\Rightarrow$ Swap out page frames onto HDD/Flash/Cloud. Swap file entry tells the operating system that the page frame is located in the swap partition. Swapped out page frames are marked in the page table. Single page frames are swapped out (\textit{frame granularity}).
		}
		\par{
			\noindent\underline{Page replacement algorithms:}
			\par{
				\noindent
				The best page frame to replace would be the one which is used the farthest in the future. But to determine this optimal page frame the operating system would need to look into the future. Thus there are several algorithms trying to approximate this behavior: clock, aging, LRU (least recently used).
			}
		}
		\par{
			\noindent\underline{Process switch:}
			\par{
				\noindent
				When the operating system (the scheduler) switches the process, e.g. the currently executed process A is put to sleep and process B is scheduled next, the page table and the program counter (PC) are switched. The MMU then uses the page table of process B and the CPU uses the program counter of process B. Figure~\ref{fig:pagetableconcurrentprocessesprocA} and~\ref{fig:pagetableconcurrentprocessesprocB} show the process switch. So each process has the illusion that it owns the whole physical address space.
			}
			\par{
				\noindent
				The process switch can be done in constant time because the operating system only needs to adjust the pointers of the MMU and the CPU, respectively. This results in a perfect isolation in memory.
			}
		}
		\clearpage
		\begin{figure}[H]
			\centering
			\begin{tikzpicture}
				\tikzset{memory/.style = {draw, minimum height = 6cm, minimum width = 2cm}}
				\tikzset{page_table/.style = {draw, minimum height = 1cm, minimum width = 0.5cm}}

				\node[memory] at (0, 0) (memory_physical) {};
				\node[left = 0 of memory_physical.north west] {\footnotesize{max}};
				\node[left = 0 of memory_physical.south west] {\footnotesize{0}};
				\node[below = 0.25 of memory_physical.south] {\footnotesize{physical memory}};
				\draw[<-, >=stealth] (memory_physical.north) -- ++(0, 0.5) node[right] (pc) {\footnotesize{PC}} node[above = 0.25] {\footnotesize{CPU}};

				\node[right = 2 of pc] (mmu) {\footnotesize{MMU}};
				\node[below = 0.5 of mmu, text width = 2.5cm, text centered] (pt) {\footnotesize{page table (of currently active process A)}};
				\draw[->, >=stealth] (mmu.south) -- (pt.north);
 
				\node[memory, right = 5 of memory_physical] (memory_procA) {};
				\node[below = 0.25 of memory_procA.south] {\footnotesize{process A}};
				\node[page_table, above = 0.25 of memory_procA.north] (page_table_procA) {}; 
				\node[right = 0 of page_table_procA.east, text width = 1cm, text centered] {\footnotesize{page table}};

				\node[memory, right = 3 of memory_procA] (memory_procB) {};
				\node[below = 0.25 of memory_procB.south] {\footnotesize{process B}};
				\node[page_table, above = 0.25 of memory_procB.north] (page_table_procB) {};
				\node[right = 0 of page_table_procB.east, text width = 1cm, text centered] {\footnotesize{page table}};

				\draw (memory_procA.east) ++(1.5, 2) -- ++(0, -1);
				\draw (memory_procA.east) ++(1.65, 2) -- ++(0, -1);

				\draw[->, >=stealth, dotted] (memory_physical.north) .. controls (memory_physical.center) and (memory_physical.east) .. (memory_procA.west);
				\draw[->, >=stealth, dotted] (pt.south) -- ++(0, -0.5) -- ++(1.5, 0) -- ++(0, 3.175) -- (page_table_procA.west);
			\end{tikzpicture}
			\caption{Page tables of concurrent processes: process A is active}
			\label{fig:pagetableconcurrentprocessesprocA}
		\end{figure}
		\begin{figure}[H]
			\centering
			\begin{tikzpicture}
				\tikzset{memory/.style = {draw, minimum height = 6cm, minimum width = 2cm}}
				\tikzset{page_table/.style = {draw, minimum height = 1cm, minimum width = 0.5cm}}

				\node[memory] at (0, 0) (memory_physical) {};
				\node[left = 0 of memory_physical.north west] {\footnotesize{max}};
				\node[left = 0 of memory_physical.south west] {\footnotesize{0}};
				\node[below = 0.25 of memory_physical.south] {\footnotesize{physical memory}};
				\draw[<-, >=stealth] (memory_physical.north) -- ++(0, 0.5) node[right] (pc) {\footnotesize{PC}} node[above = 0.25] {\footnotesize{CPU}};

				\node[right = 2 of pc] (mmu) {\footnotesize{MMU}};
				\node[below = 0.5 of mmu, text width = 2.5cm, text centered] (pt) {\footnotesize{page table (of currently active process B)}};
				\draw[->, >=stealth] (mmu.south) -- (pt.north);
 
				\node[memory, right = 5 of memory_physical] (memory_procA) {};
				\node[below = 0.25 of memory_procA.south] {\footnotesize{process A}};
				\node[page_table, above = 0.25 of memory_procA.north] (page_table_procA) {}; 
				\node[right = 0 of page_table_procA.east, text width = 1cm, text centered] {\footnotesize{page table}};

				\node[memory, right = 3 of memory_procA] (memory_procB) {};
				\node[below = 0.25 of memory_procB.south] {\footnotesize{process B}};
				\node[page_table, above = 0.25 of memory_procB.north] (page_table_procB) {};
				\node[right = 0 of page_table_procB.east, text width = 1cm, text centered] {\footnotesize{page table}};

				\draw (memory_procA.east) ++(1.5, 2) -- ++(0, -1);
				\draw (memory_procA.east) ++(1.65, 2) -- ++(0, -1);

				\draw[->, >=stealth, dotted] (memory_physical.north) .. controls (memory_physical.center) and (memory_physical.east) .. (memory_procB.west);
				\draw[->, >=stealth, dotted] (pt.south) -- ++(0, -0.5) -- ++(7, 0) -- ++(0, 3.175) -- (page_table_procB.west);
			\end{tikzpicture}
			\caption{Page tables of concurrent processes: process B is active}
			\label{fig:pagetableconcurrentprocessesprocB}
		\end{figure}
	}
}
\par{
	\noindent\underline{Kernel vs. User space}
	\par{
		\noindent
		A perfect operating system cannot crash because of a malicious user program. Is this possible? Yes, in 2009 \textit{sel4} (\textit{secure L4 $\mu$Kernel}) proved this behavior.
	}
	\par{
		\noindent
		The startup of an operating system is privileged. There exist two options to switch from user to privileged mode (in general):
		\parskip0pt\begin{enumerate}
			\item{User program is finished \& executes a privileged instruction}
			\item{A system call is executed by a user program}
		\end{enumerate}
		\begin{figure}[H]
			\centering
			\begin{tikzpicture}
				\tikzset{bubble/.style = {draw, circle, text centered}}

				\node[bubble, text width = 4cm] at (0, 0) (contextswitch) {\footnotesize{context switch is done only if a privileged instruction is executed}};
				\node[bubble, above left = 1 of contextswitch] (user) {\footnotesize{user}};
				\node[bubble, above right = 1 of contextswitch] (privileged) {\footnotesize{privileged}};
				\draw[->, >=stealth] (contextswitch.north west) -- (user.south east);
				\draw[->, >=stealth] (contextswitch.north east) -- (privileged.south west);
				\draw[<->, >=stealth, dashed] (contextswitch.north east) -- (contextswitch.north west);
				\node[left = 0.5 of user.west] (instr_t) {\footnotesize{Instructions}};
				\node[left = 0.5 of contextswitch.west] (mode_t) {\footnotesize{Mode}}; 
				\node[above = 1 of contextswitch.north] (cpu_t) {\footnotesize{CPU}};
			\end{tikzpicture}
			\caption{User and privileged mode}
			\label{fig:kernelvsuserspace}
		\end{figure}
	}
}